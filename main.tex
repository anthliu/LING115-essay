\documentclass[12pt]{article}

%
%Margin - 1 inch on all sides
%
\usepackage[letterpaper]{geometry}
\usepackage{times}
\usepackage{listings}
\usepackage{float}
\floatstyle{boxed} 
\restylefloat{figure}

\lstset
{ %Formatting for code in appendix
    basicstyle=\footnotesize,
    numbers=left,
    stepnumber=1,
    showstringspaces=false,
    tabsize=1,
    breaklines=true,
    breakatwhitespace=false,
}

\geometry{top=1.0in, bottom=1.0in, left=1.0in, right=1.0in}

%
%Doublespacing
%
\usepackage{setspace}
\doublespacing

%
%Rotating tables (e.g. sideways when too long)
%
\usepackage{rotating}


%
%Fancy-header package to modify header/page numbering (insert last name)
%
\usepackage{fancyhdr}
\pagestyle{fancy}
\lhead{} 
\chead{} 
\rhead{Liu \thepage} 
\lfoot{} 
\cfoot{} 
\rfoot{} 
\renewcommand{\headrulewidth}{0pt} 
\renewcommand{\footrulewidth}{0pt} 
%To make sure we actually have header 0.5in away from top edge
%12pt is one-sixth of an inch. Subtract this from 0.5in to get headsep value
\setlength\headsep{0.333in}

%
%Works cited environment
%(to start, use \begin{workscited...}, each entry preceded by \bibent)
% - from Ryan Alcock's MLA style file
%
\newcommand{\bibent}{\noindent \hangindent 40pt}
\newenvironment{workscited}{\newpage \begin{center} Works Cited \end{center}}{\newpage }


%
%Begin document
%
\begin{document}
\begin{flushleft}

%%%%First page name, class, etc
Anthony Liu\\
Professor Marlyse Baptista\\
LING 115\\
December 18 2017\\


%%%%Title
\begin{center}
Programming Languages and Natural Languages
\end{center}


%%%%Changes paragraph indentation to 0.5in
\setlength{\parindent}{0.5in}
%%%%Begin body of paper here

A programming language is a set of instructions that tell a computer
what actions to perform. Every application that runs on a computer
is written using a programming language.
%
If you've heard of
most popular languages used today, \textit{C, Java, Python}, etc, you may
wonder - Why don't we use natural languages such as English to
program? Surely programming in English is easier to learn and to teach
than a programming language?
%
In fact, using natural languages to create programs has been the goal
of many computer science researchers (Veres). However, the fact remains
that none of these efforts are truly using the ``full capacity" of
a natural language such as English in programming.

Consider an analogy of writing a list of instructions for baking
a cake. Though the instruction of \textit{1. Bake the cake.} may be humorous
in its lack of information, the instruction is an essentially correct
recipe of making a cake. The ambiguities of natural language
cause them to be unsuitable for programming.
%
Even the mundane \textit{x. Stir dry ingredients in bowl until mixed.}
is equally uninformative. Stir which ingredients?
What does it mean for ingredients to be mixed? How does one stir ingredients?
With a spoon? Stir clockwise? Of course, to the average cook, these questions are
irrelevant and can answered by common sense. However, when telling a
\textit{computer} to bake a cake, what does common sense mean? Does a
computer know what it means to stir ingredients?
%
By using natural language to write a recipe, we are left with
problems caused by ambiguities and a lack of precision.
Programming languages must be exact. The programmer and the computer
both agree on how a program should be executed.

\begin{figure}[h]
\centering
\caption{
``Guess the number game" in \textit{Python}.
In plain words, we can describe the execution of this
program line by line.
1. Set the \textit{answer} to be 42.
2. Let the user set the value of \textit{guess}.
3-4. While the value of \textit{guess} is not the same value
as the value of \textit{answer},
let the user set the value of \textit{guess} (the \textit{!=} word is an approximation
for the mathematical inequality symbol $\neq$.).
5. Finally, print (or display on the screen), ``You win!".
}
\begin{lstlisting}[language=Python]
answer = '42'
guess = input()
while guess != answer:
    guess = input()
print('You win!')
\end{lstlisting}
\label{fig:guess_number}
\end{figure}

Though natural language is unsuitable for direct use as a programming
language, its influence and usage in programming languages is undeniable.
In Figure \ref{fig:guess_number}, we show a simple ``Guess the number" game
in the programming language \textit{Python}.
The meaning of the program can be easily deciphered only using knowledge
of English. Knowledge that most programming languages are evaluated line
by line from top to bottom is also crucial.

The design and usage of Programming Languages is closely intertwined with spoken languages.
The similarities in the number guessing game in Figure \ref{fig:guess_number}
to its natural translation is no coincidence.
There are two reasons -
\begin{enumerate}
\item The general idea of programming languages is the same as natural
language - to put ideas into words. When a programming language needs
a word to describe an idea, that idea is often already encoded into
a word of natural language.
\item As alluded to earlier, a primary goal of programming language
designers and programmers is to have programming languages be both easy to
learn and understand. Like a large manuscript, the greatest applications
will have many subsections rewritten and modified when new features are
added or old features are broken. This need is amplified when large
teams must work on the same project, where programmers must read and
understand their respective programs to build upon it.
(As an example, as of 2014, the code
for Facebook has over 9.9 \textit{million} lines of code (Pearce).)

The most \textit{natural} way humans communicate is through natural
language. The incorporation of natural language into programming
increases the ease of understanding.
\end{enumerate}
There are two main ways natural language is incorporated into programming
- the design of the programming language and usage of the language, the
way programmer's ``speak" the language.

To speak of language design, we must revisit the problem of precision
written earlier, the difficulty of encoding our thoughts into the programs.
How do we tell a computer how to send our email?
How do we tell a computer how to display a photo on the screen?
The simple solution is to restrict our language only to \textit{simplest}
actions that a computer can do. Any application that is \textit{possible} to
create on a computer, is created by a limited set of instructions, or words.
Analogous, \textit{any} cooking recipe and technique can be described solely
by the activation and release of each muscle in the human body. The act
of stirring a bowl is carefully described from the usage of muscle groups
in the forearm, to the tightening of the fingers around the spoon.
Through this restriction, we sacrifice brevity and clarity for absolute
precision.
In the computer, these actions often include but are not limited to
adding numbers, saving numbers into registers, or memory locations,
and ``jumping", which can be used to repeat already executed code \footnotemark.

\begin{figure}[h]
\centering
\caption[Caption for LOF]{
Sample of the \textit{MIPS} ISA.
For example, when the computer reads the number 40, the computer will store
a value into its register, or memory. \footnotemark
}
\begin{tabular}{l l r}
Instruction & Mnemonic & Encoding \\
\hline
Load Byte & LB & 32 \\
Store Byte & SB & 40 \\
Add & ADD & 0 \\
\end{tabular}

\label{fig:MIPS_sample}
\end{figure}

These simple sets of instructions are called
\textit{Instruction Set Architectures}, or \textit{ISA}'s. Most ISA's will
contain less than 100 instructions. A sample of an ISA still used today is shown
in Figure \ref{fig:MIPS_sample}. Languages that directly 

\newpage

\begin{center}
Notes
\end{center}

1. You may wonder how it is possible a finite set of simple instructions
can be used to construct useful programs. In fact, it was mathematically proven
by computer scientist Alan Turing that any program that can be written on a computer, can be written using
a simple \textit{Turing machine} with (technically) three instructions, move,
read, and write (Turing).

2. In practice, computers will read in binary format, a format of representing
numbers using only two symbols, 1 and 0.

\setlength{\parindent}{0.5in}


%%%%Works cited
\begin{workscited}

\bibent
Veres, Sandor M., and J. Patrik Adolfsson. "A natural language programming solution for executable papers." \textit{Procedia Computer Science} 4 (2011): 678-687.

\bibent
Pearce, James. "9.9 million lines of code and still moving fast - Facebook open source ...." https://code.facebook.com/posts/292625127566143/9-9-million-lines-of-code-and-still-moving-fast-facebook-open-source-in-2014/. Accessed 17 Dec. 2017.

\bibent
Turing, Alan M. "Computing machinery and intelligence." Mind 59.236 (1950): 433-460.

\end{workscited}

\end{flushleft}
\end{document}
\}
